\documentclass[a4paper,titlepage,dvipdfmx]{jarticle}
\usepackage{caption}
% 数式用
\usepackage{longtable}
\usepackage{amsmath, amsfonts}  % 数式用
\usepackage{bm} % 数式中の太字
\usepackage{amssymb} % 記号
% 画像用
\usepackage{float} % 画像の挿入箇所を固定
\usepackage[dvipdfmx]{graphicx} % 画像挿入用
\usepackage{wrapfig} % 画像の周りに本文を流し込み
% ページレイアウト
\usepackage{here} % 画像を強制的に出力
\usepackage{url} % URLリンク
\usepackage{hyperref} % 
\hypersetup{
  colorlinks=false, % リンクに色をつけない設定
  bookmarks=true, % 以下ブックマークに関する設定
  bookmarksnumbered=true, % ブックマークに節番号などをつけるか
  pdfborder={0 0 0}, % 枠なし
  bookmarkstype=toc % 目次情報のブックマークを作る
}
\usepackage{multirow} % 表で行結合
\usepackage[margin=20truemm]{geometry} %余白調整
% 文字装飾
\usepackage{color} % 文字色
\usepackage[table,xcdraw]{xcolor} % 表の色
\usepackage{ascmac} % 丸枠
\usepackage{fancybox} % 丸枠
\usepackage{booktabs} % 表の横線を調整するためのパッケージ
\usepackage{tabularx} % 表の幅を調整するためのパッケージ
\usepackage{tcolorbox} %ボックスやフレームを作成するためのパッケージ
% プログラム用
\definecolor{comment}{rgb}{0.52,0.60,0.00} %green
\usepackage{listings,jvlisting} %日本語のコメントアウトをする場合jvlisting(もしくはjlisting)が必要
%ここからソースコードの表示に関する設定
\lstset{
  basicstyle=\ttfamily,
  showstringspaces=false,
  commentstyle=\color{comment},
  keywordstyle= \color{blue}\bfseries,
  xleftmargin=0zw, %左余白
  xrightmargin=0zw,%右余白
  frame=single,%枠線
  frameround=tttt,%枠線角丸
}
\usepackage{fancyhdr} % ヘッダー・フッター
\pagestyle{fancy}
    % \lhead{専門チャレンジ(PCプログラミング)} %ヘッダ左
    \chead{第3回 計算} %ヘッダ中央
    \rhead{授業資料} %ヘッダ右
    \cfoot{\thepage} %フッタ中央.ページ番号を表示
    \rfoot{\today} %フッタ右
    \renewcommand{\headrulewidth}{0.4pt} %ヘッダの線の太さ
    \renewcommand{\footrulewidth}{0.4pt} %フッタの線の太さ
    
\begin{document}
\section{はじめに}
\textbf{lesson2}では、
\textbf{変数}について学習しました。
今回は、変数を用いて計算を行う方法について学習します。
\section{計算}
\subsection{演算子}
\begin{table}[H]
  \centering
  \caption{演算子一覧}
  \begin{tabular}{|c|c|c|c|c|c|}
    \hline
    記号 & 演算     & 説明         & 例      & 結果 \\ \hline
    +  & 加算     & 足し算        & 1 + 2  & 3  \\ \hline
    -  & 減算     & 引き算        & 3 - 2  & 1  \\ \hline
    *  & 乗算     & 掛け算        & 2 * 3  & 6  \\ \hline
    /  & 除算     & 割り算        & 6 / 3  & 2  \\ \hline
    \% & 剰余     & 割り算の余り     & 5 \% 2 & 1  \\ \hline
    // & 切り捨て除算 & 割り算の商を切り捨て & 5 // 2 & 2  \\ \hline
    ** & 累乗     & べき乗        & 2 ** 3 & 8  \\ \hline
  \end{tabular}
\end{table}
\subsection{変数を使った計算}
\begin{lstlisting}[caption=変数を使った計算,label=変数を使った計算]
a = 1
b = 2
c = a + b
print(c)
\end{lstlisting}
\begin{itembox}[l]{結果}
  3
\end{itembox}
以上のように演算子を用いることで、変数を使った計算が可能です。
文字列の場合は、\texttt{+}演算子を用いることで文字列の連結が可能です。
\begin{lstlisting}[caption=文字列の連結,label=文字列の連結]
  hello = "Hello"
  world = "World"
  sentence = hello + world
  print(sentence)
  
\end{lstlisting}
\begin{itembox}[l]{結果}
  HelloWorld
\end{itembox}
また、文字列と\texttt{*}演算子を用いることで、文字列の繰り返しが可能です。
\begin{lstlisting}[caption=文字列の繰り返し,label=文字列の繰り返し]
  hello = "Hello"
  sentence = hello * 3
  print(sentence)
\end{lstlisting}
\begin{itembox}[l]{結果}
  HelloHelloHello
\end{itembox}
\section{実践}
\subsection{フィボナッチ数列}
\begin{itembox}[l]{問題}
  以下のプログラムを作成し、実行してください。\\
  $f_{0}=0$, $f_{1}=1$とし、$f_{n+2}=f_{n}+f_{n+1}$となる数列はフィボナッチ数列と呼ばれます。
  このフィボナッチ数列の4番目の数を求めるプログラムを作成してください。
\end{itembox}
\begin{itembox}[l]{ヒント}
  まず、わかっている$f_{0}$と$f_{1}$を変数に代入します。
  この二変数を用いると、$f_{2}$を求めることができます。
  これを繰り返していくことで、フィボナッチ数列の4番目の数を求めることができます。
\end{itembox}
\subsection{文字列の操作}
文字列型にはいくつかの便利な\textbf{関数}が用意されています。
今回は、\texttt{replace()}関数と\texttt{len()}関数を用います。
\begin{itembox}[l]{replace()関数}
  \texttt{replace()}関数は、文字列の中で指定した文字列を別の文字列に置き換えます。
  \begin{lstlisting}[caption=replace()関数,label=replace()関数]
    sentence = "Hello, World!"
    new_sentence = sentence.replace("World", "Python")
    print(new_sentence)
  \end{lstlisting}
  ``World''という文字列が、``Python''に置き換わります。
  つまり、\texttt{replace()}関数の中で一番最初に指定した文字が
  二番目に指定した文字に置き換わります。
  \begin{itembox}[l]{結果}
    Hello, Python!
  \end{itembox}
\end{itembox}
\begin{itembox}[l]{len()関数}
  \texttt{len()}関数は、文字列の長さを返します。
  \begin{lstlisting}[caption=len()関数,label=len()関数]
    sentence = "Hello, World!"
    length = len(sentence)
    print(length)
  \end{lstlisting}
  文字列の長さは、空白や記号も含めて数えられます。
  \begin{itembox}[l]{結果}
    13
  \end{itembox}
\end{itembox}
以上を踏まえて、以下の問題に取り組んでください。
\begin{itembox}[l]{問題}
  以下のプログラムを作成し、実行してください。\\
  文字列``Hello, World!''の``World''を``Python''に置き換え、
  その後の文字列の長さを求めるプログラムを作成してください。
\end{itembox}
\end{document}