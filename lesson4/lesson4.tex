\documentclass[a4paper,titlepage,dvipdfmx]{jarticle}
\usepackage{caption}
% 数式用
\usepackage{longtable}
\usepackage{amsmath, amsfonts}  % 数式用
\usepackage{bm} % 数式中の太字
\usepackage{amssymb} % 記号
% 画像用
\usepackage{float} % 画像の挿入箇所を固定
\usepackage[dvipdfmx]{graphicx} % 画像挿入用
\usepackage{wrapfig} % 画像の周りに本文を流し込み
% ページレイアウト
\usepackage{here} % 画像を強制的に出力
\usepackage{url} % URLリンク
\usepackage{hyperref} % 
\hypersetup{
  colorlinks=false, % リンクに色をつけない設定
  bookmarks=true, % 以下ブックマークに関する設定
  bookmarksnumbered=true, % ブックマークに節番号などをつけるか
  pdfborder={0 0 0}, % 枠なし
  bookmarkstype=toc % 目次情報のブックマークを作る
}
\usepackage{multirow} % 表で行結合
\usepackage[margin=20truemm]{geometry} %余白調整
% 文字装飾
\usepackage{color} % 文字色
\usepackage[table,xcdraw]{xcolor} % 表の色
\usepackage{ascmac} % 丸枠
\usepackage{fancybox} % 丸枠
\usepackage{booktabs} % 表の横線を調整するためのパッケージ
\usepackage{tabularx} % 表の幅を調整するためのパッケージ
\usepackage{tcolorbox} %ボックスやフレームを作成するためのパッケージ
% プログラム用
\definecolor{comment}{rgb}{0.52,0.60,0.00} %green
\usepackage{listings,jvlisting} %日本語のコメントアウトをする場合jvlisting(もしくはjlisting)が必要
%ここからソースコードの表示に関する設定
\lstset{
  basicstyle=\ttfamily,
  showstringspaces=false,
  commentstyle=\color{comment},
  keywordstyle= \color{blue}\bfseries,
  xleftmargin=0zw, %左余白
  xrightmargin=0zw,%右余白
  frame=single,%枠線
  frameround=tttt,%枠線角丸
}
\usepackage{fancyhdr} % ヘッダー・フッター
\pagestyle{fancy}
    % \lhead{専門チャレンジ(PCプログラミング)} %ヘッダ左
    \chead{第4回 リスト、辞書} %ヘッダ中央
    \rhead{授業資料} %ヘッダ右
    \cfoot{\thepage} %フッタ中央.ページ番号を表示
    \rfoot{\today} %フッタ右
    \renewcommand{\headrulewidth}{0.4pt} %ヘッダの線の太さ
    \renewcommand{\footrulewidth}{0.4pt} %フッタの線の太さ
    
\begin{document}
\section{はじめに}
\textbf{lesson3}では、
\textbf{計算}について学習しました。\\
今回は、データをまとめて管理する際に便利な
\textbf{リスト}と\textbf{辞書}について学習します。
\section{リスト}
\subsection{リストとは}
\textbf{リスト}は、複数のデータをまとめて管理するためのデータ型です。\\
リストは、複数のデータをカンマで区切り、角括弧で囲んで表現します。\\
\begin{lstlisting}[caption=リストの例,label=リストの例]
# リストの例
list1 = [1, 2, 3, 4, 5]
print(list1)
\end{lstlisting}
\begin{lstlisting}[caption=リストの例の実行結果,label=リストの例の実行結果]
# 実行結果
[1, 2, 3, 4, 5]
\end{lstlisting}
\subsection{リストの操作}
\subsubsection{リストの要素へのアクセス}
リストの要素にアクセスするには、\textbf{何番目かを表す}インデックスを指定します。\\
リストは、\textbf{0番目}から始まります。\\
今回の例で言うと、\texttt{list1}の0番目は1、1番目は2、2番目は3、3番目は4、4番目は5です。\\
\begin{lstlisting}[caption=リストの要素へのアクセス,label=リストの要素へのアクセス]
# リストの要素へのアクセス
list1 = [1, 2, 3, 4, 5]
print(list1[0]) # 1
print(list1[1]) # 2
print(list1[2]) # 3
print(list1[3]) # 4
print(list1[4]) # 5
\end{lstlisting}

\subsubsection{リストの要素の変更}
リストの要素を変更するには、\textbf{インデックス}を指定して、新しい値を代入します。\\
\begin{lstlisting}[caption=リストの要素の変更,label=リストの要素の変更]
# リストの要素の変更
list1 = [1, 2, 3, 4, 5]
list1[0] = 10
print(list1) # [10, 2, 3, 4, 5]
\end{lstlisting}

\subsubsection{リストの要素の追加}
リストの末尾に要素を追加するには、\texttt{append()}メソッドを使用します。\\
\begin{lstlisting}[caption=リストの要素の追加,label=リストの要素の追加]
# リストの要素の追加
list1 = [1, 2, 3, 4, 5]
list1.append(6)
print(list1) # [1, 2, 3, 4, 5, 6]
\end{lstlisting}

\subsubsection{リストの要素の削除}
リストの要素を削除するには、\texttt{del}文を使用します。\\
\begin{lstlisting}[caption=リストの要素の削除,label=リストの要素の削除]
# リストの要素の削除
list1 = [1, 2, 3, 4, 5]
del list1[0]
print(list1) # [2, 3, 4, 5]
\end{lstlisting}

\subsubsection{リストの要素の挿入}
リストの任意の位置に要素を挿入するには、\texttt{insert()}メソッドを使用します。\\
\begin{lstlisting}[caption=リストの要素の挿入,label=リストの要素の挿入]
# リストの要素の挿入
list1 = [1, 2, 3, 4, 5]
list1.insert(0, 0)
print(list1) # [0, 1, 2, 3, 4, 5]
\end{lstlisting}

\subsubsection{リストの要素の削除(値指定)}
リストの要素を削除するには、\texttt{remove()}メソッドを使用します。\\
注意すべき点として、\texttt{remove()}メソッドは、インデックスを指定しているのではなく
\textbf{値を指定}して削除します。\\
\begin{lstlisting}[caption=リストの要素の削除(値指定),label=リストの要素の削除(値指定)]
# リストの要素の削除(値指定)
list1 = [1, 2, 3, 4, 5]
list1.remove(3)
print(list1) # [1, 2, 4, 5]
\end{lstlisting}

\section{辞書}
\subsection{辞書とは}
\textbf{辞書}は、キーと値をペアで管理するためのデータ型です。\\
\textbf{リスト}がインデックスでデータにアクセスしているのに対し、
\textbf{辞書}はキーでデータにアクセスします。\\
辞書は、\textbf{波括弧}で囲んで表現します。\\
\begin{lstlisting}[caption=辞書の例,label=辞書の例]
# 辞書の例
dict1 = {"name": "Alice", "age": 14, "birthday": "2010-04-01"}
print(dict1)
\end{lstlisting}
\begin{lstlisting}[caption=辞書の例の実行結果,label=辞書の例の実行結果]
# 実行結果
{'name': 'Alice', 'age': 14, 'birthday': '2010-04-01'}
\end{lstlisting}

辞書の操作にはいくつかの方法がありますが、
今回は、\textbf{キーを指定して値を取得}する方法についてのみ学習します。

\subsection{辞書の操作}
\subsubsection{辞書の要素へのアクセス}
辞書の要素にアクセスするには、\textbf{キー}を指定します。\\
\begin{lstlisting}[caption=辞書の要素へのアクセス,label=辞書の要素へのアクセス]
# 辞書の要素へのアクセス
dict1 = {"name": "Alice", "age": 14, "birthday": "2010-04-01"}
print(dict1["name"]) # Alice
print(dict1["age"]) # 14
print(dict1["birthday"]) # 2010-04-01
\end{lstlisting}

要素の変更、削除については、
\textbf{リスト}と同様の方法で行います。

\section{練習問題}
\begin{itembox}
  変数\texttt{list1}に\texttt{[1, 2, 3, 4, 5]}を代入し、
  以下の操作を行いなさい。
  \begin{enumerate}
    \item リストの要素を全て表示しなさい。
    \item リストの0番目の要素を表示しなさい。
    \item リストの4番目の要素を表示しなさい。
    \item リストの0番目の要素を10に変更し、全ての要素を表示しなさい。
    \item リストの末尾に6を追加し、全ての要素を表示しなさい。
    \item リストの0番目の要素を削除し、全ての要素を表示しなさい。
    \item リストの3番目に0を挿入し、全ての要素を表示しなさい。
    \item リストの3を削除し、全ての要素を表示しなさい。
  \end{enumerate}
\end{itembox}
\begin{itembox}
  変数\texttt{dict1}に\texttt{\{"name": "Alice", "age": 14, "birthday": "2010-04-01"\}}を代入し、
  以下の操作を行いなさい。
  \begin{enumerate}
    \item 辞書の要素を全て表示しなさい。
    \item 辞書の\texttt{"name"}の要素を表示しなさい。
    \item 辞書の\texttt{"age"}の要素を表示しなさい。
    \item 辞書の\texttt{"birthday"}の要素を表示しなさい。
  \end{enumerate}
\end{itembox}
\end{document}