\documentclass[a4paper,titlepage,dvipdfmx]{jarticle}
\usepackage{caption}
% 数式用
\usepackage{longtable}
\usepackage{amsmath, amsfonts}  % 数式用
\usepackage{bm} % 数式中の太字
\usepackage{amssymb} % 記号
% 画像用
\usepackage{float} % 画像の挿入箇所を固定
\usepackage[dvipdfmx]{graphicx} % 画像挿入用
\usepackage{wrapfig} % 画像の周りに本文を流し込み
% ページレイアウト
\usepackage{here} % 画像を強制的に出力
\usepackage{url} % URLリンク
\usepackage{hyperref} % 
\hypersetup{
  colorlinks=false, % リンクに色をつけない設定
  bookmarks=true, % 以下ブックマークに関する設定
  bookmarksnumbered=true, % ブックマークに節番号などをつけるか
  pdfborder={0 0 0}, % 枠なし
  bookmarkstype=toc % 目次情報のブックマークを作る
}
\usepackage{multirow} % 表で行結合
\usepackage[margin=20truemm]{geometry} %余白調整
% 文字装飾
\usepackage{color} % 文字色
\usepackage[table,xcdraw]{xcolor} % 表の色
\usepackage{ascmac} % 丸枠
\usepackage{fancybox} % 丸枠
\usepackage{booktabs} % 表の横線を調整するためのパッケージ
\usepackage{tabularx} % 表の幅を調整するためのパッケージ
\usepackage{tcolorbox} %ボックスやフレームを作成するためのパッケージ
% プログラム用
\definecolor{comment}{rgb}{0.52,0.60,0.00} %green
\usepackage{listings,jvlisting} %日本語のコメントアウトをする場合jvlisting(もしくはjlisting)が必要
%ここからソースコードの表示に関する設定
\lstset{
  basicstyle={\ttfamily},
  identifierstyle={\small},
  commentstyle={\smallitshape},
  keywordstyle={\small\bfseries},
  ndkeywordstyle={\small},
  stringstyle={\small\ttfamily},
  frame={tb},
  breaklines=true,
  columns=[l]{fullflexible},
  numbers=left,
  xrightmargin=0zw,
  xleftmargin=3zw,
  numberstyle={\scriptsize},
  stepnumber=1,
  numbersep=1zw,
  lineskip=-0.5ex,
  }
\usepackage{fancyhdr} % ヘッダー・フッター
\renewcommand{\lstlistingname}{ソースコード}
\pagestyle{fancy}
    % \lhead{専門チャレンジ(PCプログラミング)} %ヘッダ左
    \chead{第6回 繰り返し} %ヘッダ中央
    \rhead{授業資料} %ヘッダ右
    \cfoot{\thepage} %フッタ中央.ページ番号を表示
    \rfoot{\today} %フッタ右
    \renewcommand{\headrulewidth}{0.4pt} %ヘッダの線の太さ
    \renewcommand{\footrulewidth}{0.4pt} %フッタの線の太さ
    
\begin{document}
\section{はじめに}
\textbf{lesson5}では、
条件分岐について学習しました。
今回の授業では、繰り返しについて学習します。
たくさんのデータを処理する際に、繰り返しを使うことで、
効率的にプログラムを書くことができます。

\section{繰り返し}
\subsection{for文}
\textbf{for文}は、指定した回数だけ処理を繰り返す制御構造です。
\begin{lstlisting}[caption=for文の構文]
  for [variable] in [sequence]:
    [process]
\end{lstlisting}
\textbf{variable}には、繰り返し処理で使用する変数を指定します。
\textbf{sequence}には、リストやrange()関数を指定します。
\textbf{process}には、繰り返し処理を記述します。

\begin{lstlisting}[caption=5回Hello,World!を表示する例,label=helloloop]
  for i in range(5):
    print("Hello, World!")
\end{lstlisting}

\begin{lstlisting}[caption=5回足し算を行う例,label=plusloop]
  ans = 0
  for i in range(5):
    ans = ans + i
  print(ans) # 10
\end{lstlisting}

\begin{lstlisting}[caption=リストの中身を表示する例,label=listloop]
  my_list = ["apple", "banana", "cherry"]
  for i in my_list:
    print(i)
\end{lstlisting}

\subsection{break}
\textbf{break}は、繰り返し処理を途中で終了する制御構造です。
\begin{lstlisting}[caption=breakの例,label=break]
  for i in range(10):
    if i == 5:
      break
    print(i)
\end{lstlisting}
上のソースコードは、本来であれば0から9まで表示されるはずですが、
繰り返しが行われる中で、\texttt{i}が5になると
途中の\texttt{if}文に反応し、\texttt{break}が実行されます。
つまり、この繰り返しでは、0から4までの数字が表示されます。

\subsection{continue}
\textbf{continue}は、繰り返し処理の途中で、次の繰り返し処理に移る制御構造です。
\begin{lstlisting}[caption=continueの例,label=continue]
  for i in range(10):
    if i == 5:
      continue
    print(i)
\end{lstlisting}
上のソースコードは、本来であれば0から9まで順に表示されるはずですが、
繰り返しが行われる中で、\texttt{i}が5になると
途中の\texttt{if}文に反応し、\texttt{continue}が実行されます。
つまり、この繰り返しでは、5だけ表示されません。

\subsection{while文}
\textbf{while文}は、条件が真の間、処理を繰り返す制御構造です。
\begin{lstlisting}[caption=while文の構文]
  while [condition]:
    [process]
\end{lstlisting}
主な使われ方としては無限ループとしての利用があります。
\begin{lstlisting}[caption=無限ループの例,label=infinite]
  while True:
    print("Hello, World!")
\end{lstlisting}
上記のコードは、強制終了しない限り、無限に``Hello, World!''を表示し続けます。

\begin{lstlisting}[caption=while文の例,label=while]
  i = 0
  while i < 5:
    print(i)
    i += 1  
\end{lstlisting}
\texttt{i}が5より小さい間、\texttt{i}を表示し、\texttt{i}に1を加えるという処理を繰り返します。
つまり、\texttt{i}が5になった時点で繰り返し処理が終了します。

\section{練習問題}
\begin{enumerate}
  \item \texttt{for}文を使って、1から10までの数字を表示するプログラムを作成してください。
  \item \texttt{for}文を使って、1から10までの数字の合計を求めるプログラムを作成してください。
  \item \texttt{for}文を使って、リスト\texttt{[1, 2, 3, 4, 5]}の中身を表示するプログラムを作成してください。
  \item \texttt{for}文を使って、リスト\texttt{[1, 2, 3, 4, 5]}の中身を合計するプログラムを作成してください。
  \item \texttt{while}文を使って、1から10までの数字を表示するプログラムを作成してください。
  \item \texttt{while}文を使って、1から10までの数字の合計を求めるプログラムを作成してください。
  \item \texttt{while}文を使って、リスト\texttt{[1, 2, 3, 4, 5]}の中身を表示するプログラムを作成してください。
\end{enumerate}

\subsection{入力が正しいかどうかを判定するプログラム}
次回の授業で自動販売機のプログラムを作成します。
そこでユーザーからの入力を受け付けるプログラムを作成します。
\begin{lstlisting}[caption=入力が正しいかどうかを判定するプログラム,label=check]
while True: # 無限ループ
  num = input(" 数字を入力してください: ") # ユーザーからの入力を受け付ける
  if num.isdigit(): # 入力が数字かどうかを判定
    break # 数字の場合は繰り返しを終了
  elif num == "exit":
    print(" 終了します")
    exit() # プログラムを終了する関数
  else:
    print(" 数字を入力してください")
print(num)
\end{lstlisting}
上記のプログラムは、ユーザーからの入力が数字かどうかを判定し、数字であればその値を表示します。
数字でない場合であるかつ、``exit''と入力された場合は、プログラムを終了します。
それ以外の場合は、``数字を入力してください''と表示します。
\end{document}