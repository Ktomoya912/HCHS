\documentclass[a4paper,titlepage,dvipdfmx]{jarticle}
\usepackage{caption}
% 数式用
\usepackage{longtable}
\usepackage{amsmath, amsfonts}  % 数式用
\usepackage{bm} % 数式中の太字
\usepackage{amssymb} % 記号
% 画像用
\usepackage{float} % 画像の挿入箇所を固定
\usepackage[dvipdfmx]{graphicx} % 画像挿入用
\usepackage{wrapfig} % 画像の周りに本文を流し込み
% ページレイアウト
\usepackage{here} % 画像を強制的に出力
\usepackage{url} % URLリンク
\usepackage{hyperref} % 
\hypersetup{
  colorlinks=false, % リンクに色をつけない設定
  bookmarks=true, % 以下ブックマークに関する設定
  bookmarksnumbered=true, % ブックマークに節番号などをつけるか
  pdfborder={0 0 0}, % 枠なし
  bookmarkstype=toc % 目次情報のブックマークを作る
}
\usepackage{multirow} % 表で行結合
\usepackage[margin=20truemm]{geometry} %余白調整
% 文字装飾
\usepackage{color} % 文字色
\usepackage[table,xcdraw]{xcolor} % 表の色
\usepackage{ascmac} % 丸枠
\usepackage{fancybox} % 丸枠
\usepackage{booktabs} % 表の横線を調整するためのパッケージ
\usepackage{tabularx} % 表の幅を調整するためのパッケージ
\usepackage{tcolorbox} %ボックスやフレームを作成するためのパッケージ
% プログラム用
\definecolor{comment}{rgb}{0.52,0.60,0.00} %green
\usepackage{listings,jvlisting} %日本語のコメントアウトをする場合jvlisting(もしくはjlisting)が必要
%ここからソースコードの表示に関する設定
\lstset{
  basicstyle={\ttfamily},
  identifierstyle={\small},
  commentstyle={\smallitshape},
  keywordstyle={\small\bfseries},
  ndkeywordstyle={\small},
  stringstyle={\small\ttfamily},
  frame={tb},
  breaklines=true,
  columns=[l]{fullflexible},
  numbers=left,
  xrightmargin=0zw,
  xleftmargin=3zw,
  numberstyle={\scriptsize},
  stepnumber=1,
  numbersep=1zw,
  lineskip=-0.5ex,
  }
\usepackage{fancyhdr} % ヘッダー・フッター
\renewcommand{\lstlistingname}{ソースコード}
\pagestyle{fancy}
    % \lhead{専門チャレンジ(PCプログラミング)} %ヘッダ左
    \chead{第10回 関数} %ヘッダ中央
    \rhead{授業資料} %ヘッダ右
    \cfoot{\thepage} %フッタ中央.ページ番号を表示
    \rfoot{\today} %フッタ右
    \renewcommand{\headrulewidth}{0.4pt} %ヘッダの線の太さ
    \renewcommand{\footrulewidth}{0.4pt} %フッタの線の太さ
    
\begin{document}

\section{はじめに}
みなさんがなんとなく使っていた\texttt{print()}や\texttt{input()}というものは、
「関数」と呼ばれるものです。

この関数を呼び出すことによって、関数それぞれが提供している機能を利用することができます。
また、新たに自分独自の関数を作成し、利用することもできます。
今回の授業では、自分で関数を定義し、利用してもらいます。
\section{関数の定義方法}
関数の定義方法はいたって簡単です。
\begin{lstlisting}[caption=関数の定義,label=関数の定義]
def 関数名( 引数 ):
  process
  return 戻り値
\end{lstlisting}
\texttt{print()}関数とあてはめながらそれぞれについて解説していきます。

\subsection*{関数名}
関数名は、その関数が何をするかを表す名前です。
\texttt{print()}関数でいうなれば、関数名は\texttt{print}です。
\subsection*{引数}
引数は、関数に渡す値のことです。
\texttt{print()}関数でいうなれば、引数は表示したい文字列や数字です。
\subsection*{処理}
関数が行う処理です。
\texttt{print()}関数内では、引数に渡された値を画面に表示する処理が行われます。
\subsection*{戻り値}
関数が返す値のことです。
\texttt{print()}関数は、引数に渡された値を表示するだけなので、戻り値はありません。

\section{関数の例}
二つの値の足し算を行って、その結果を返す関数を定義してみましょう。
\begin{lstlisting}[caption=関数の例,label=関数の例]
def add(a, b):
  return a + b

result = add(3, 5)
print(result)
\end{lstlisting}
\texttt{add()}という関数は、引数\texttt{a}と\texttt{b}を受け取り、その和を返します。
今回では、\texttt{a}に3、\texttt{b}に5が渡されているので、\texttt{add()}関数は8を返します。

\section{関数の利点}
今までの授業では、自分で関数を定義せずに一つ一つの機能を列挙して、実装していました。
しかし、関数を使うことで、同じ機能を使いたいというときに、その機能をまとめておくことができます。
また、関数を使うことで、プログラムの見通しを良くすることができます。

\section{課題}
\subsection*{課題1}
関数に二つの引数、\texttt{a}と\texttt{b}を受け取り、その差を返す関数\texttt{sub()}を定義してください。
\begin{lstlisting}[caption=課題1,label=課題1]
# ここに関数sub()を定義してください

result = sub(10, 3)
print(result) # 7
\end{lstlisting}

\subsection*{課題2}
関数に二つの引数、\texttt{text}と\texttt{times}を受け取り、\texttt{text}を\texttt{times}回繰り返した文字列を返す関数\texttt{repeat()}を定義してください。
\begin{lstlisting}[caption=課題2,label=課題2]
# ここに関数repeat()を定義してください

result = repeat("Hello", 3)
print(result) # HelloHelloHello
\end{lstlisting}
\subsubsection*{ヒント}
この問題には解法がいくつかあります。
\begin{itemize}
  \item \texttt{for}文を使って\texttt{times}回繰り返す
  \item \texttt{text}を\texttt{times}倍する
\end{itemize}
一番上の方法は単純に行うと改行が挟まれてしまうため、\texttt{print()}関数に、\texttt{end=""}という引数を渡すことで改行を抑制することができます。

\subsection*{応用課題}
フィボナッチ数列を求める関数\texttt{fib()}を定義してください。
引数には、求めたいフィボナッチ数列の項数を受け取ります。

\subsubsection*{フィボナッチ数列の式}
フィボナッチ数列は以下の式で求めることができます。
\begin{align}
  F(0) & = 0                                \\
  F(1) & = 1                                \\
  F(n) & = F(n-1) + F(n-2) \quad (n \geq 2)
\end{align}
\begin{lstlisting}[caption=応用課題,label=応用課題]
# ここに関数fib()を定義してください

result = fib(10)
print(result) # 55
\end{lstlisting}
\subsubsection*{ヒント}
関数の返り値に関数自身を使うことで、再帰的に関数を呼び出すことができます。
\begin{lstlisting}[caption=再帰関数,label=再帰関数]
def func(n):
  print(n)
  if n == 0:
    return 0
  return func(n-1)
\end{lstlisting}
この関数を外側で引数10を与えて呼び出すと、
\begin{lstlisting}[caption=再帰関数の呼び出し,label=再帰関数の呼び出し]
func(10)
# 10
# 9
# 8
...
# 0
\end{lstlisting}
というように、10から0までの数字が順番に表示されます。

\end{document}